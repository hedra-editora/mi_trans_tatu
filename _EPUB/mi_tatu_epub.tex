\title{A mulher que virou tatu}

\title{Yuxabu yaixni}


Eliane Camargo (\textit{organização})

Anita Ekman (\textit{ilustração})

2ª edição

\chapter{}

\textbf{edição brasileira©} Hedra 2022\\
\textbf{organização} Eliane Camargo\\
\textbf{ilustração©} Anita Ekman

\textbf{coordenação da coleção} Luísa Valentini\\
\textbf{edição} Jorge Sallum\\
\textbf{coedição} Suzana Salama\\
\textbf{assistência editorial} Paulo Henrique Pompermaier\\
\textbf{revisão} Renier Silva\\
\textbf{capa} Lucas Kroëff\\

\textbf{\textsc{isbn}} 978-65-89705-72-7

\textbf{conselho editorial} Adriano Scatolin, Antonio Valverde, Caio Gagliardi, Jorge Sallum, Ricardo Valle, Tales Ab'Saber, Tâmis Parron
 
\bigskip
\textit{Grafia atualizada segundo o Acordo Ortográfico da Língua
Portuguesa de 1990, em vigor no Brasil desde 2009.}\\

\vfill
\textit{Direitos reservados em língua\\ 
portuguesa somente para o Brasil}

\textsc{editora hedra ltda.}\\
R.~Fradique Coutinho, 1139 (subsolo)\\
05416--011 São Paulo \textsc{sp} Brasil\\
Telefone/Fax +55 11 3097 8304

editora@hedra.com.br\\
www.hedra.com.br

Foi feito o depósito legal.

\chapter{}

\textbf{A mulher que virou tatu} é uma história originalmente registrada no início do século \textsc{xx}, pelo historiador João Capistrano de Abreu. A língua hoje não é mais escrita do modo como Capistrano a registrou, portanto esta edição é atualizada --- além de bilíngue e ilustrada. Fala de dois alimentos que os Caxinauá plantam: o milho, mais apreciado e de cultivo mais difícil, e a batata doce, mais rápido e simples. Na história, a família de uma velha lhe dá batata doce por dar menos trabalho de produzir, já que os idosos Caxinauá não trabalham e sua alimentação deve ser garantida pelos parentes.

\textbf{Eliane Camargo} é etnolinguista. É doutora em Linguística Descritiva pela Universidade de Paris (Sorbonne) e estuda a língua e cultura de três grupos indígenas, os Caxinauá, os Aparai e os Wayana. Coordenou, entre 2006 e 2011, a divisão etnolinguística do projeto de documentação franco-alemão da cultura e língua Caxinauá do \textsc{dobes}.

\textbf{Anita Ekman} é artista visual, \textit{performer} e ilustradora que trabalha com as artes ameríndias e afro-brasileiras. Como especialista em arte indígena, trabalhou na formação da coleção \textit{Great Masters of Popular Art in Ibero-America}, do Banamex Cultural Fund.

\textbf{Capistrano de Abreu} (1953--1927) foi historiador, mas produziu também dentro dos campos da etnografia e linguística. Em 1914 registrou, pela primeira vez, a língua e o modo de vida Caxinauá junto a dois jovens provenientes do povo, do rio Ibuaçu. Esse trabalho deu origem ao livro \textit{Hantxa huni kuin}, sobre a língua dos Caxinauá do rio Ibuaçu, afluente do Muru.

\textbf{Mundo Indígena} reúne materiais produzidos com pensadores de diferentes povos indígenas e pessoas que pesquisam, trabalham ou lutam pela garantia de seus direitos. Os livros foram feitos para serem utilizados pelas comunidades envolvidas na sua produção, e por isso uma parte significativa das obras é bilíngue. Esperamos divulgar a imensa diversidade linguística dos povos indígenas no Brasil, que compreende mais de 150 línguas pertencentes a mais de trinta famílias linguísticas.


\chapter{Apresentação}

Esta história fala de dois alimentos que os Caxinauá cultivam em seus roçados, o milho e a batata doce. A batata doce é uma raiz tuberosa e de fácil cultivo: plantando uma só vez é possível colher muitas vezes, pela propagação de suas ramas. Já o cultivo do milho é mais trabalhoso: a cada vez que se colhe, é preciso esperar a época de plantio para plantar as sementes, que leva cerca de seis meses para colher novamente. Os Caxinauás apreciam e consomem mais o milho, mas também plantam a batata doce, por crescer rápido e dar pouco trabalho.

Os idosos caxinauá não trabalham no roçado; sua alimentação deve ser garantida pelos seus genros e por sua família em geral. Na história, a família da velha lhe dá batata doce
por dar menos trabalho de produzir.

Ao comer o milho verde, que é mais macio, a família, sobretudo o genro, reclama por ela não deixar o milho amadurecer, o que a deixa triste e a leva a querer virar tatu.

\section{Quem são os Caxinauá}

A família linguística pano é composta por cerca de trinta grupos, espalhados em uma vasta região transfronteiriça entre a Bolívia, o Brasil e o Peru.

Os quase oito mil Caxinauá fazem parte desta família, ocupando a fronteira entre o Brasil
e o Peru. No Brasil, eles vivem em 12 terras indígenas e, no Peru, eles ocupam todo o rio Curanja e uma parte do rio Purus --- da cidade de Puerto Esperanza até a embocadura
do rio Curanja.

No Peru, e na região do rio Purus, no Peru e no Brasil, as mulheres e crianças falam apenas a língua caxinauá. Nas demais regiões elas já são bilíngues e, em alguns locais,
monolíngues em português.

Todas as comunidades têm escola formal, onde são somente alfabetizados em caxinauá. O
restante do ensino é ministrado em espanhol, no Peru, e em português, no Brasil. O material escolar em língua caxinauá é escasso no Peru, e corrente no Brasil.

\chapter{Como foi feito este livro}

No início do século \textsc{xx}, o historiador João Capistrano de Abreu trabalhou
com dois jovens caxinauás provenientes do rio Ibuaçu, afluente do rio Muru, por sua vez afluente do rio Tarauacá, na bacia do rio Juruá, no estado do Acre.

Com a venda de borracha de sua região, estes jovens foram levados para Manaus. Lá eles conheceram Luís Sombra, amigo de Capistrano de Abreu, que os encaminhou ao historiador, cada um de uma vez, para o Rio de Janeiro, onde ficaram na casa de Capistrano e trabalharam
com ele no registro da sua língua e de seu modo de vida.

Esse trabalho deu origem ao livro \textit{Hantxa huni kuin, a língua dos caxinauás do rio Ibuaçu, afluente do muru} (prefeitura de Tarauacá). O livro foi publicado pela primeira vez em 1914.

Hoje em dia, a língua caxinauá não é escrita do modo que Capistrano a
registrou, e os próprios caxinauá não conseguem ler esses relatos de
cem anos atrás. Além disso, a língua ainda não era muito estudada, então
a tradução proposta por Capistrano era muito entrecortada e inicial.


Pensando que essas histórias poderiam ser lidas hoje em uma forma mais acessível tanto aos caxinauá quanto aos falantes de português, a linguista Eliane Camargo, que
trabalha com eles desde 1987, resolveu revisar o livro e refazer a tradução, dentro do programa de documentação de cultura e língua caxinauá, \textsc{dobes}, financiado pela
Fundação Volkswagen.

Esta história é uma parte dessa versão revisada por Eliane, que consideramos ser interessante para crianças e para adultos e, por isso, publicamos neste livrinho. Uma parte
dos direitos autorais recebidos com a publicação do livro será destinada à realização de oficinas de língua e cultura onde os caxinauá continuarão pensando novos modos de
escrever e apresentar sua língua e sua cultura em suas próprias escolas e para pessoas de outros lugares.


\chapter{Para ler as palavras caxinauá}

A língua caxinauá apresenta quatro vogais --- \textit{a}, \textit{e}, \textit{i}, \textit{u} --- e catorze consoantes --- \textit{b}, \textit{d}, \textit{h}, \textit{k}, \textit{m}, \textit{n}, \textit{p}, \textit{s}, \textit{x}, \textit{t}, \textit{ts}, \textit{tx}, \textit{w}, \textit{y}. Notem que \textit{a} ordem do \textit{x} na sequência do alfabeto muda;
ele aparece logo após o \textit{s}.

Nesta língua há três sons não existentes em português:

\begin{itemize}
\item A vogal \textit{e} que é um \textit{schwa}, ou seja, um \textit{e} pronunciado com \textit{a} língua plana e o som sai de trás. Este som é comum em inglês, em francês e em muitas línguas da amazônia;

\item A consoante \textit{ts} requer uma pronuncia em um só som, \textit{t}\,+\,\textit{s};

\item A consoante \textit{x} é uma retroflexa, isto é, a massa da língua vai para trás e a ponta dela toca ligeiramente o palato. Este som é comum em chinês;

\item Sequência consonântica \textit{t}\,+\,\textit{x} (\textit{tch}) é comum em espanhol, grafado \textit{ch}.
\end{itemize}

As palavras dissilábicas são muito comuns:

\begin{itemize}
\item \textit{Baka}, como em \textit{peixe};
\item \textit{Hiwe}, como em \textit{casa};
\item \textit{Kene}, como em \textit{grafismo};
\item \textit{Tapu}, como em \textit{jirau}, \textit{ponte}.
\end{itemize}

Mas há palavras de uma só sílaba: \textit{hi}, ``árvore''; ou de mais sílabas: \textit{taka}--da, ``galinha''; \textit{bepukudu}, ``borboleta''.


\chapter{A mulher que virou tatu}

\chapter{}

Quando a família se reunia, só se comia\\
batata doce. Faziam roçado e plantavam\\
batata doce. Só davam batata doce bichada\\
para a velha comer. É o que davam à velha.\\
Ela vivia com a família.

\textit{Kadi besti pikin, itxa wani kiaki.\\
Bai wakin hawen ni katsidan, kadi banaaki.\\
Xena besti pimiski hawen pitimaken.\\
Yuxabudan eskani kiaki.Hawen\\
nabube hiwea.}

\chapter{}

A família dela fazia roçado e tinha\\
um milharal. A velha desdentada\\
não podia comer milho seco.

\textit{Hawen nabu bai waxun, xeki\\
banaimabu. Yuxabudan xeta uma,\\
haska waxun piti, kuxi pitima.}

\chapter{}

Quando a velha vivia com\\
a família, desperdiçava-se\\
muito milho verde.\\
Ela queria virar tatu, pois\\
não podia comer o milho verde,\\
visto que a família lhe dizia:\\
--- Ô, velha, você só fica\\
comendo o nosso milho verde.\\
Ela respondia:\\
--- Como só milho verde, por\\
não poder comer milho seco.\\
Não tenho dente.\\
A mulher respondeu isso\\
e ficou pensando no que\\
a família lhe disse.

\textit{Hawen nabube hiwea,\\
mawa xeki pati txakaaya.\\
Yuxabudan yaix katsidan\\
eskani kiaki. Haska waxun,\\
pitima, xeki patxi besti piaya,\\
hawen nabun itxaa:\\
--- Yuxabun, min en xeki\\
patxi besti piai, aka.\\
Yuxabu yuikin:\\
--- En haska waxun piti kuxi\\
pitima. En xeta uma, en xeta\\
umabin.\\
Ainbun yuia, ainbu\\
ninkaxun.}

\chapter{}

Então ela foi sozinha mata\\
adentro e, ao voltar, à\\
tardezinha, disse à sua filha:\\
--- Filha, eu vou virar tatu. Sou\\
desdentada e por isso não posso\\
comer milho verde. Vou embora.\\
A filha respondeu-lhe:\\
--- Mamãe, é por isso que\\
você não pode comer?\\
--- Minha filha, é. É por isso que\\
não posso comer, respondeu-lhe.\\
A filha replicou:\\
--- Mamãe, então coma\\
só milho verde!

\textit{Hanunkain, yuxabu ni medan\\
ha mesti kaa, badi kaaya\\
huxun, hawen bake yuia:\\
--- En bake, eadan en yaixi kaai. En\\
xeta uma. Haska waxun,\\
piti kuxi pitima, en ikai, aka.\\
Hawen bake yuikin:\\
--- En ewan, min haska waxun\\
pitimamen, aka. En bake, en haska\\
waxun, pitimabin, aka.\\
Hawen bake yuia:\\
--- Ewan, xeki patxi\\
besti piwe, aka.}

\chapter{}

A velha só comia milho\\
verde por não poder comer\\
milho seco, que é duro.\\
Quando acabou o milho verde\\
do roçado deles, os homens\\
estavam zangados\\
e lhe disseram:\\
--- Velha, você acabou com o\\
nosso roçado de milho verde.

\textit{Yuxabun xeki patxi besti piaya.\\
Haska waxun, piti kuxi pitima.\\
Hatun bai xeki patxi keyun waaya,\\
hunibun sinaxun, yuxabu yuikin:\\
--- Yuxabun, min en xeki\\
patxi bai keyuna, aka.}


\chapter{}

Ela lhes respondeu:\\
--- É por não poder comer milho\\
seco. Sou desdentada. Por\\
sinal, minha filha me disse:\\
--- Mamãe, coma milho verde!\\
E respondi:\\
--- Vou comer, sim.\\
Assim disse a velha. Mas os\\
homens retrucaram-lhe:\\
--- Pare de comer o nosso milho!

\textit{Yuxabu yuikin:\\
--- En haska waxun, piti kuxi\\
pitima, en ikai, aka. Eadan,\\
en xeta umabin, aka.\\
Habia en baken:\\
--- Xeki patxi piwe, ewan,\\
yui.\\
En piai, aka. Yuxabun haska waa.\\
Hunibun yuxabu yuikin:\\
--- En xeki ea keyunyamawe, aka.}

\chapter{}

Não podendo mais comer milho verde,\\
a velha chorou e quis virar tatu.\\
Foi sozinha para o mato e cavou um buraco.

\textit{Yuxabu haska waa, ana hawa pitima,\\
kaxaaya. Yuxabu yaixi ka katsi eskani kiaki.\\
Ha mesti ni medan kaxun, kini waaya.}

\chapter{}

Um homem que havia ido caçar a viu\\
cavando o buraco, aproximou-se\\
dela e lhe perguntou:\\
--- Ei, velha, por que você\\
está cavando um buraco?\\
--- É porque não posso comer milho seco.\\
Só posso comer milho verde. Mas como\\
esculhambaram comigo, vim cavar um\\
buraco para ser tatu, respondeu.\\
O homem a escutou e ficou pensativo,\\
chorando tristemente.

\textit{Huni piaya kaxun, yuxabun kini\\
waa, betxia, hunin yuxabu yukaa:\\
--- Yuxabun, min hawa\\
katsi kini waai? aka.\\
--- En haska waxun, piti kuxi\\
pitima, xeki patxi besti en piaya,\\
ea itxabu, huxun, en kini waai\\
yaix katsidan, aka.\\
Hunin ninkaa, hawen\\
dabanen iki, kaxaaya.}

\chapter{}

Ao regressar, pergunto\\
à família dela:\\
--- Por que vocês\\
esculhambaram com a velha?\\
--- Esculhambei porque ela só\\
comia o milho verde do meu\\
roçado. Eu a insultei e\\
ela foi embora.\\
--- A velha foi para lá cavar\\
buraco, eu a vi. Ela quer virar\\
tatu — disse o caçador.

\textit{Haska wabidani, hukidan,\\
hawen nabu yuia:\\
--- En nabun, mi hawa\\
katsi yuxabu\\
itxa kamen, aka.\\
--- Habia en xeki patxi\\
ea pianaya,\\
en itxaa, kaaki, aka.\\
--- Yuxabudan uani kini\\
waai, en uinbidanxuki,\\
yaix katsidan, aka.}

\chapter{}

O caçador disse ao seu\\
filho que estava chorando:\\
--- A velha que vocês esculhambaram\\
já virou tatu. Ela já tem rabo, casco\\
nas costas, casco na cabeça. Virou\\
todinha tatu. A velha sente falta\\
do filho.\\ 
\textit{Vou buscá-lo}, disse a\\
si mesma. Chamou por ele, gritando\\
\textit{ruu}, fazendo barulho de tatu.

\textit{Hawen bake yuia, kaxaaya.\\
--- Yuxabu ma yaixa. Hanunkain, hinayatan,\\
pexakayatan, nuxakayatan, buxakayatan.\\
Haska wakin, keyua. Yuxabu hawen bake\\
manui:\\ 
\textit{En bake itannun} ika. \textit{Huu} aka.}

\chapter{}

O seu filho pequeno sentia falta\\
da mãe, e chorava sem parar.\\
Ele andava sozinho, chorando,\\
de um lado para o outro.\\
A velha ouviu o choro e pensou:\\
--- O meu filho está chorando, vou vê-lo.\\
Voltou à aldeia para vê-lo; lá estava ele\\
sentado, chorando. Quando viu o tatu,\\
alegrou-se, e o tatu lhe disse:\\
--- Meu filho, eu vou te levar.

\textit{Hawen bake hawen ibu\\
manui, kaxawankainkainaya.\\
Hawen bake, ha mesti bai\\
tanai, kaxakukuaya.\\
Yuxabu kaxai ninkaa:\\
--- En bake kaxaai, uintannun, ika.\\
Huaya, bake pixta kaxai, tsauken,\\
bake pixta yaix betxia, benimaaya.\\
Yaixin bake pixta yuikin:\\
--- En bake, en mia yuai, aka.}

\chapter{}

A criança que estava\\
sentada ficou contente.\\
Então a velha levou o menino\\
para morar dentro do buraco.\\
Ela lhe fez o rabo, o casco das\\
costas, o casco da cabeça.\\
E a criança ficou feliz.\\
A velha havia feito a mesma\\
coisa para virar tatu.

\textit{Bake pixta benimaai, tsauken.\\
Hanunkain, yuxabun bake pixta\\
hawen hiwe medan yukin.\\
Bake pixta hina waxun, pexaka\\
waxun, buxaka waxun.\\
Haska waxun, bake pixta\\
benimani kiaki.\\
Yuxabudan eskani kiaki,\\
yaix katsidan.}

\chapter{}

A história diz que quem\\
domesticou a batata doce para\\
podermos comer foi o tatu, e\\
quando não tinha batata doce\\
para comer, o tatu comia minhoca.\\
Foi assim que a velha fez para\\
virar tatu, transformou o corpo\\
e passou a comer batata doce e,\\
quando não tinha, comia minhoca.

\textit{Kadi bikindan, yaixin bini kiaki.\\
Kadimakendan yaixdan\\
xena besti pimis kiaki.\\
Yuxabudan eskani kiaki, yaix\\
katsidan. Hatixunki, yamaki.}

