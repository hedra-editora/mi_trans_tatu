\chapter[Para ler as palavras caxinauá]{Para ler as palavras\break caxinauá}

A língua caxinauá apresenta quatro vogais --- \textit{a}, \textit{e}, \textit{i}, \textit{u} --- e catorze consoantes --- \textit{b}, \textit{d}, \textit{h}, \textit{k}, \textit{m}, \textit{n}, \textit{p}, \textit{s}, \textit{x}, \textit{t}, \textit{ts}, \textit{tx}, \textit{w}, \textit{y}. Notem que \textit{a} ordem do \textit{x} na sequência do alfabeto muda;
ele aparece logo após o \textit{s}.

Nesta língua há três sons não existentes em português:

\begin{itemize}
\item A vogal \textit{e} que é um \textit{schwa}, ou seja, um \textit{e} pronunciado com \textit{a} língua plana e o som sai de trás. Este som é comum em inglês, em francês e em muitas línguas da amazônia;

\item A consoante \textit{ts} requer uma pronuncia em um só som, \textit{t}\,+\,\textit{s};

\item A consoante \textit{x} é uma retroflexa, isto é, a massa da língua vai para trás e a ponta dela toca ligeiramente o palato. Este som é comum em chinês;

\item Sequência consonântica \textit{t}\,+\,\textit{x} (\textit{tch}) é comum em espanhol, grafado \textit{ch}.
\end{itemize}

As palavras dissilábicas são muito comuns:

\begin{itemize}
\item Baka, como em \textit{peixe};
\item Hiwe, como em \textit{casa};
\item Kene, como em \textit{grafismo};
\item Tapu, como em \textit{jirau}, \textit{ponte}.
\end{itemize}

Mas há palavras de uma só sílaba: \textit{hi}, ``árvore''; ou de mais sílabas: \textit{taka}--da, ``galinha''; \textit{bepukudu}, ``borboleta''.


