\chapter{Para ler as palavras caxinauá}


A língua caxinauá apresenta quatro vogais (a, e, i, u) e catorze consoantes (b, d, h, k, m, n, p, s, x, t, ts, tx, w, y). Notem que a ordem do ‘x’ na sequência do alfabeto muda;
ele aparece logo após o ‘s’.

Nesta língua há três sons não existentes em português:

\begin{itemize}
\item A vogal ‘e’ que é um schwa, ou seja, um ‘e’ pronunciado com a língua plana e o som sai de trás. Este som é comum em inglês, em francês e em muitas línguas da amazônia.

\item A consoante ‘ts’ requer uma pronuncia em um só som, t+s. 

\item A consoante ‘x’ é uma retroflexa, isto é, a massa da língua vai para trás e a ponta dela toca ligeiramente o palato. Este som é comum em chinês.

\item Sequência consonântica t+x (tch) é comum em espanhol, grafado ‘ch’.
\end{itemize}

As palavras dissilábicas são muito comuns:

\begin{itemize}

\item Baka, ‘peixe’;

\item Hiwe, ‘casa’;

\item Kene, ‘grafismo’;

\item Tapu, ‘jirau’, ‘ponte’.
\end{itemize}

Mas há palavras de uma só sílaba: hi, ‘árvore’; ou de mais sílabas: taka- da, ‘galinha’, bepukudu, ‘borboleta’.


