\chapter{Como foi feito este livro}

No início do século \textsc{xx}, o historiador João Capistrano de Abreu trabalhou
com dois jovens caxinauás provenientes do rio Ibuaçu, afluente do rio Muru, por sua vez afluente do rio Tarauacá, na bacia do rio Juruá, no estado do Acre.

Com a venda de borracha de sua região, estes jovens foram levados para Manaus. Lá eles conheceram Luís Sombra, amigo de Capistrano de Abreu, que os encaminhou ao historiador, cada um de uma vez, para o Rio de Janeiro, onde ficaram na casa de Capistrano e trabalharam
com ele no registro da sua língua e de seu modo de vida.

Esse trabalho deu origem ao livro \textit{Hantxa huni kuin, a língua dos caxinauás do rio Ibuaçu, afluente do muru} (prefeitura de Tarauacá). O livro foi publicado pela primeira vez em 1914.

Hoje em dia, a língua caxinauá não é escrita do modo que Capistrano a
registrou, e os próprios caxinauá não conseguem ler esses relatos de
cem anos atrás. Além disso, a língua ainda não era muito estudada, então
a tradução proposta por Capistrano era muito entrecortada e inicial.


\textls[-20]{Pensando que essas histórias poderiam ser lidas hoje em uma forma mais acessível tanto aos caxinauá quanto aos falantes de português, a linguista Eliane Camargo, que
trabalha com eles desde 1987, resolveu revisar o livro e refazer a tradução, dentro do programa de documentação de cultura e língua caxinauá, \textsc{dobes}, financiado pela
Fundação Volkswagen.}\looseness=-1

Esta história é uma parte dessa versão revisada por Eliane, que consideramos ser interessante para crianças e para adultos e, por isso, publicamos neste livrinho. Uma parte
dos direitos autorais recebidos com a publicação do livro será destinada à realização de oficinas de língua e cultura onde os caxinauá continuarão pensando novos modos de
escrever e apresentar sua língua e sua cultura em suas próprias escolas e para pessoas de outros lugares.


