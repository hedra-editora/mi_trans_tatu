\textbf{A mulher que virou tatu} Os quase oito mil Caxinauá fazem parte da família linguística pano, composta por cerca de trinta grupos, e ocupam a fronteira entre o Brasil e o Peru. No Brasil, vivem em doze terras indígenas e, no Peru, ocupam todo o rio Curanja e uma parte do rio Purus --- da cidade de Puerto Esperanza até a embocadura do rio Curanja. O historiador João Capistrano de Abreu foi quem, no início do século XX, registrou pela primeira vez a língua e o modo de vida Caxinauá junto a dois jovens provenientes da etnia, do rio Ibuaçu.

Esse trabalho deu origem ao livro \textit{Hantxa huni kuin} de 1914, sobre a língua dos Caxinauá do rio Ibuaçu, afluente do Muru. Hoje a língua não é mais escrita do modo que Capistrano a registrou, e os próprios Caxinauá não conseguem ler esses relatos de cem anos atrás. Para tornar a história deste povo acessível, a linguista Eliane Camargo, que trabalha com eles desde 1987, revisou e refez a tradução da publicação já ultrapassada de Capistrano. Esta edição do excerto da tradução atualizada pela linguista, bilíngue, conta também com diversas ilustrações.

\textbf{Eliane Camargo} \lipsum[2]

\textbf{Anita Ekman} é artista visual, performer e ilustradora que trabalha com as artes ameríndias e afro-brasileiras. Como especialista em arte indígena, trabalhou na formação da coleção \textit{Great Masters of Popular Art in Ibero-America}, do Banamex Cultural Fund.

\textbf{Capistrano de Abreu} \lipsum[2]

\textbf{Coleção Mundo Indígena} reúne materiais produzidos com pensadores de diferentes povos indígenas e pessoas que pesquisam, trabalham ou lutam pela garantia de seus direitos. Os livros foram feitos para serem utilizados pelas comunidades envolvidas na sua produção, e por isso uma parte significativa das obras é bilíngue. Esperamos divulgar a imensa diversidade linguística dos povos indígenas no Brasil, que compreende mais de 150 línguas pertencentes a mais de trinta famílias linguísticas.



