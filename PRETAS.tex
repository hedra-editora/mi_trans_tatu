\textbf{A mulher que virou tatu} é uma história originalmente registrada pelo historiador João Capistrano de Abreu, no início do século \textsc{xx}. Hoje a língua não é mais escrita do modo como Capistrano a registrou, portanto esta edição é atualizada --- além de bilíngue e ilustrada. Fala de dois alimentos que os Caxinauá cultivam: o milho, mais apreciado e de cultivo mais difícil, e a batata doce, mais rápido e simples. Na história, a família da velha lhe dá batata doce por dar menos trabalho de produzir, já que os idosos Caxinauá não trabalham e sua alimentação deve ser garantida pelos parentes.

\textbf{Eliane Camargo} \lipsum[2]

\textbf{Anita Ekman} é artista visual, performer e ilustradora que trabalha com as artes ameríndias e afro-brasileiras. Como especialista em arte indígena, trabalhou na formação da coleção \textit{Great Masters of Popular Art in Ibero-America}, do Banamex Cultural Fund.

\textbf{Capistrano de Abreu} \lipsum[2]

\textbf{Coleção Mundo Indígena} reúne materiais produzidos com pensadores de diferentes povos indígenas e pessoas que pesquisam, trabalham ou lutam pela garantia de seus direitos. Os livros foram feitos para serem utilizados pelas comunidades envolvidas na sua produção, e por isso uma parte significativa das obras é bilíngue. Esperamos divulgar a imensa diversidade linguística dos povos indígenas no Brasil, que compreende mais de 150 línguas pertencentes a mais de trinta famílias linguísticas.



