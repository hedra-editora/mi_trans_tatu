\textbf{A mulher que virou tatu} \textls[10]{é uma história originalmente registrada no início do século \textsc{xx}, pelo historiador João Capistrano de Abreu. A língua hoje não é mais escrita do modo como Capistrano a registrou, portanto esta edição é atualizada --- além de bilíngue e ilustrada. Fala de dois alimentos que os Caxinauá plantam: o milho, mais apreciado e de cultivo mais difícil, e a batata doce, mais rápido e simples. Na história, a família de uma velha lhe dá batata doce por dar menos trabalho de produzir, já que os idosos Caxinauá não trabalham e sua alimentação deve ser garantida pelos parentes.}

\textbf{Eliane Camargo} é etnolinguista. É doutora em Linguística Descritiva pela Universidade de Paris (Sorbonne) e estuda a língua e cultura de três grupos indígenas, os Caxinauá, os Aparai e os Wayana. Coordenou, entre 2006 e 2011, a divisão etnolinguística do projeto de documentação franco-alemão da cultura e língua Caxinauá do \textsc{dobes}.

\textbf{Anita Ekman} é artista visual, \textit{performer} e ilustradora que trabalha com as artes ameríndias e afro-brasileiras. Como especialista em arte indígena, trabalhou na formação da coleção \textit{Great Masters of Popular Art in Ibero-America}, do Banamex Cultural Fund.

\textbf{Capistrano de Abreu} \textls[20]{(1953--1927) foi historiador, mas produziu também dentro dos campos da etnografia e linguística. Em 1914 registrou, pela primeira vez, a língua e o modo de vida Caxinauá junto a dois jovens provenientes do povo, do rio Ibuaçu. Esse trabalho deu origem ao livro \textit{Hantxa huni kuin}, sobre a língua dos Caxinauá do rio Ibuaçu, afluente do Muru.}

\textls[25]{\textbf{Mundo Indígena} reúne materiais produzidos com pensadores de diferentes povos indígenas e pessoas que pesquisam, trabalham ou lutam pela garantia de seus direitos. Os livros foram feitos para serem utilizados pelas comunidades envolvidas na sua produção, e por isso uma parte significativa das obras é bilíngue. Esperamos divulgar a imensa diversidade linguística dos povos indígenas no Brasil, que compreende mais de 150 línguas pertencentes a mais de trinta famílias linguísticas.}


