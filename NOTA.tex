\chapter{Nota da organizadora}

\textls[10]{Esta história fala de dois alimentos que os Caxinauá cultivam em seus roçados, o milho e a batata doce. A batata doce é uma raiz tuberosa e de fácil cultivo: plantando uma só vez é possível colher muitas vezes, pela propagação de suas ramas. Já o cultivo do milho é mais trabalhoso: a cada vez que se colhe, é preciso esperar a época de plantio para plantar as sementes, que leva cerca de seis meses para colher novamente. Os Caxinauá apreciam e consomem mais o milho, mas também plantam a batata doce, por crescer rápido e dar pouco trabalho.}

Os idosos caxinauá não trabalham no roçado; sua alimentação deve ser garantida pelos seus genros e por sua família em geral. Na história, a família da velha lhe dá batata doce
por dar menos trabalho de produzir.

\textls[-10]{Ao comer o milho verde, que é mais macio, a família, sobretudo o genro, reclama por ela não deixar o milho amadurecer, o que a deixa triste e a leva a querer virar tatu.}

\section{quem são os caxinauá}

\textls[20]{A família linguística pano é composta por cerca de trinta grupos, espalhados em uma vasta região transfronteiriça entre a Bolívia, o Brasil e o Peru.}

Os quase oito mil Caxinauá fazem parte desta família, ocupando a fronteira entre o Brasil
e o Peru. No Brasil, eles vivem em 12 terras indígenas e, no Peru, eles ocupam todo o rio Curanja e uma parte do rio Purus --- da cidade de Puerto Esperanza até a embocadura
do rio Curanja.

\textls[20]{No Peru, e na região do rio Purus, no Peru e no Brasil, as mulheres e crianças falam apenas a língua caxinauá. Nas demais regiões elas já são bilíngues e, em alguns locais,
monolíngues em português.}

\textls[20]{Todas as comunidades têm escola formal, onde são somente alfabetizados em caxinauá. O
restante do ensino é ministrado em espanhol, no Peru, e em português, no Brasil. O material escolar em língua caxinauá é escasso no Peru, e corrente no Brasil.}