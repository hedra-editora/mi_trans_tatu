\newcommand{\linha}[2]{\ifdef{#2}{\linhalayout{#1}{#2}}{}}

\begingroup\tiny
\parindent=0cm
\thispagestyle{empty}

\textbf{copyright}\quad			 {Hedra \the\year}\\ \textit{Direitos cedidos à Casa de Letras Eireli}\\

%\textbf{edição consultada}\quad			 {Transcrição feita por Capistrano de Abreu}\\
\textbf{organização\,©}\quad		 			 {Eliane Camargo}\\
\textbf{ilustrações\,©}\quad			 		 {Anita Ekman}\\
\textbf{coordenação da coleção}\quad		 {Luísa Valentini}\\

\textbf{edição}\quad			 			 {Jorge Sallum}\\
\textbf{coedição}\quad			 			 {Suzana Salama}\\
\textbf{assistência editorial}\quad			 {Paulo Henrique Pompermaier}\\		
\textbf{capa}\quad			 				 {Lucas Kroëff}\\

\textbf{\textsc{isbn}}\quad			 		 {978-65-6011-156-1}

\vfill

\begin{minipage}{7cm}
\textbf{Dados Internacionais de Catalogação na Publicação (\textsc{cip})\\
(Câmara Brasileira do Livro, \textsc{sp}, Brasil)}

\textbf{\hrule}\smallskip

Popyguá, Timóteo Verá Tupã\\

\textit{A mulher que virou tatu}. Apresentação de Anita Ekman; posfácio de Freg J.\,Stokes. 1.\,ed. São Paulo, \textsc{sp}: Hedra, 2024.\\

\textsc{isbn} 978-85-7715-964-2\\

1. Literatura indígena 2. Povos indígenas (Guarani): usos e costumes\\ 
\textsc{i}. Ekman, Anita \textsc{ii}. Stokes, Freg J. \textsc{iii}. Título \textsc{iv}. \textit{Ka'a miri'i}\\

24--222844 \hfill \textsc{cdd}: 980.41

\textbf{\hrule}\smallskip

\textbf{Elaborado por Eliane de Freitas Leite (\textsc{crb} 8/\,8415)}\\

\textbf{Índices para catálogo sistemático:}\\
1. Povos indígenas: Brasil (980.41)

\end{minipage}

\vfill

\textit{Grafia atualizada segundo o Acordo Ortográfico da Língua\\
Portuguesa de 1990, em vigor no Brasil desde 2009.}\\

\textit{Direitos reservados em língua\\
portuguesa somente para o Brasil}\\

\textsc{casa de letras eireli}\\
Rua Fradique Coutinho, 1139\\
05416--001 São Paulo \textsc{sp} Brasil\\
Telefone +55 11 3914 7790\\\smallskip
comercial@casadeletras.com.br\\
www.casadeletras.com.br\\
\smallskip

Foi feito o depósito legal.

\endgroup
\pagebreak