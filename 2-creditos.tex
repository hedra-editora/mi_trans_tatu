\newcommand{\linha}[2]{\ifdef{#2}{\linhalayout{#1}{#2}}{}}

\begingroup\tiny
\parindent=0cm
\thispagestyle{empty}

\textbf{edição brasileira©}\quad			 {Hedra \the\year}\\
%\textbf{edição consultada}\quad			 {Transcrição feita por Capistrano de Abreu}\\
\textbf{organização\,©}\quad		 			 {Eliane Camargo}\\
\textbf{ilustrações\,©}\quad			 		 {Anita Ekman}\\
\textbf{coordenação da coleção}\quad		 {Luísa Valentini}\\

\textbf{edição}\quad			 			 {Jorge Sallum}\\
\textbf{coedição}\quad			 			 {Suzana Salama}\\
\textbf{assistência editorial}\quad			 {Paulo Henrique Pompermaier}\\		
\textbf{capa}\quad			 				 {Lucas Kroëff}\\

\textbf{\textsc{isbn}}\quad			 		 {978-65-89705-72-7}

\vfill

\begin{minipage}{7cm}
\textbf{Dados Internacionais de Catalogação na Publicação (\textsc{cip})\\
(Câmara Brasileira do Livro, \textsc{sp}, Brasil)}

\textbf{\hrule}\smallskip

Camargo, Eliane\\

\textit{A mulher que virou tatu}. Eliane Camargo; apresentação de Anita Ekman; posfácio de Freg J.\,Stokes. 2.\,ed. São Paulo, \textsc{sp}: Hedra, 2025.\\


\textsc{isbn} 978-65-89705-72-7\\


1.\,Conto. 2.\,Literatura brasileira. \textsc{i}.\,Camargo, Eliane. \textsc{ii}.\,Ekman, Anita. \textsc{iii}.\,Título.\\


\hfill \textsc{cdd}: 869.93


\textbf{\hrule}\smallskip

\textbf{Elaborado por Janaina Ramos (\textsc{crb} 8/\,9166)}\\

\textbf{Índices para catálogo sistemático:}\\
\textsc{i}.\,Conto : Literatura brasileira


\end{minipage}

\vfill

\textit{Grafia atualizada segundo o Acordo Ortográfico da Língua\\
Portuguesa de 1990, em vigor no Brasil desde 2009.}\\

\textit{Direitos reservados em língua\\
portuguesa somente para o Brasil}\\

\textsc{editora hedra ltda.}\\
Av.~São Luís, 187, Piso 3, Loja 8 (Galeria Metrópole)\\
01046--912 São Paulo \textsc{sp} Brasil\\
Telefone/Fax +55 11 3097 8304\\\smallskip
editora@hedra.com.br\\
www.hedra.com.br\\
\smallskip

Foi feito o depósito legal.

\endgroup
\pagebreak