\chapter{A mulher que virou tatu}

Quando a família se reunia, só se comia
batata doce. Faziam roçado e plantavam
batata doce. Só davam batata doce bichada
para a velha comer. É o que davam à velha.
Ela vivia com a família.

A família dela fazia roçado e tinha
um milharal. A velha desdentada
não podia comer milho seco.

Quando a velha vivia com
a família, desperdiçava-se
muito milho verde.
Ela queria virar tatu, pois
não podia comer o milho verde,
visto que a família lhe dizia:
— Ô, velha, você só fica
comendo o nosso milho verde.
Ela respondia:
— Como só milho verde, por
não poder comer milho seco.
Não tenho dente.
A mulher respondeu isso
e ficou pensando no que
a família lhe disse.

Então ela foi sozinha mata
adentro e, ao voltar, à
tardezinha, disse à sua filha:
— Filha, eu vou virar tatu. Sou
desdentada e por isso não posso
comer milho verde. Vou embora.
A filha respondeu-lhe:
— Mamãe, é por isso que
você não pode comer?
— Minha filha, é. É por isso que
não posso comer — respondeu-lhe.
A filha replicou:
— Mamãe, então coma
só milho verde!

A velha só comia milho
verde por não poder comer
milho seco, que é duro.
Quando acabou o milho verde
do roçado deles, os homens
estavam zangados
e lhe disseram:
— Velha, você acabou com o
nosso roçado de milho verde.
Ela lhes respondeu:
— É por não poder comer milho
seco. Sou desdentada. Por
sinal, minha filha me disse:
“Mamãe, coma milho verde!”, e
respondi “Vou comer, sim”.
Assim disse a velha. Mas os
homens retrucaram-lhe:
— Pare de comer o nosso milho!

Não podendo mais comer milho verde,
a velha chorou e quis virar tatu.
Foi sozinha para o mato e cavou um buraco.

Um homem que havia ido caçar a viu
cavando o buraco, aproximou-se
dela e lhe perguntou:
— Ei, velha, por que você
está cavando um buraco?
— É porque não posso comer milho seco.
Só posso comer milho verde. Mas como
esculhambaram comigo, vim cavar um
buraco para ser tatu, respondeu.
O homem a escutou e ficou pensativo,
chorando tristemente.

Ao regressar, pergunto
à família dela:
— Por que vocês
esculhambaram com a velha?
— Esculhambei porque ela só
comia o milho verde do meu
roçado. Eu a insultei e
ela foi embora.
— A velha foi para lá cavar
buraco, eu a vi. Ela quer virar
tatu — disse o caçador.

O caçador disse ao seu
filho que estava chorando:
— A velha que vocês esculhambaram
já virou tatu. Ela já tem rabo, casco
nas costas, casco na cabeça. Virou
todinha tatu. A velha sente falta
do filho. “Vou buscá-lo”, disse a
si mesma. Chamou por ele, gritando
“ruu”, fazendo barulho de tatu.

O seu filho pequeno sentia falta
da mãe, e chorava sem parar.
Ele andava sozinho, chorando,
de um lado para o outro.
A velha ouviu o choro e pensou:
— O meu filho está chorando, vou vê-lo.
Voltou à aldeia para vê-lo; lá estava ele
sentado, chorando. Quando viu o tatu,
alegrou-se, e o tatu lhe disse:
— Meu filho, eu vou te levar.

A criança que estava
sentada ficou contente.
Então a velha levou o menino
para morar dentro do buraco.
Ela lhe fez o rabo, o casco das
costas, o casco da cabeça.
E a criança ficou feliz.
A velha havia feito a mesma
coisa para virar tatu.

A história diz que quem
domesticou a batata doce para
podermos comer foi o tatu, e
quando não tinha batata doce
para comer, o tatu comia minhoca.
Foi assim que a velha fez para
virar tatu, transformou o corpo
e passou a comer batata doce e,
quando não tinha, comia minhoca.

\chapter{Yuxabu yaixni}

Kadi besti pikin, itxa wani kiaki.
Bai wakin hawen ni katsidan, kadi banaaki.
Xena besti pimiski hawen pitimaken.
Yuxabudan eskani kiaki.Hawen
nabube hiwea.


Hawen nabu bai waxun, xeki
banaimabu. Yuxabudan xeta uma,
haska waxun piti, kuxi pitima.

Hawen nabube hiwea,
mawa xeki pati txakaaya.
Yuxabudan yaix katsidan
eskani kiaki. Haska waxun,
pitima, xeki patxi besti piaya,
hawen nabun itxaa:
— Yuxabun, min en xeki
patxi besti piai, aka.
Yuxabu yuikin:
– En haska waxun piti kuxi
pitima. En xeta uma, en xeta
umabin. Ainbun yuia, ainbu
ninkaxun.

Hanunkain, yuxabu ni medan
ha mesti kaa, badi kaaya
huxun, hawen bake yuia:
– En bake, eadan en yaixi kaai. En
xeta uma. Haska waxun,
piti kuxi pitima, en ikai, aka.
Hawen bake yuikin:
— En ewan, min haska waxun
pitimamen, aka. En bake, en haska
waxun, pitimabin, aka.
Hawen bake yuia:
— Ewan, xeki patxi
besti piwe, aka.

Yuxabun xeki patxi besti piaya.
Haska waxun, piti kuxi pitima.
Hatun bai xeki patxi keyun waaya,
hunibun sinaxun, yuxabu yuikin:
— Yuxabun, min en xeki
patxi bai keyuna, aka.
Yuxabu yuikin:
— En haska waxun, piti kuxi
pitima, en ikai, aka. Eadan,
en xeta umabin, aka.
Habia en baken:
— Xeki patxi piwe, ewan,
yui. En piai, aka.
Yuxabun haska waa.
Hunibun yuxabu yuikin:
— En xeki ea keyunyamawe, aka.

Yuxabu haska waa, ana hawa pitima,
kaxaaya. Yuxabu yaixi ka katsi eskani kiaki.
Ha mesti ni medan kaxun, kini waaya.

Huni piaya kaxun, yuxabun kini
waa, betxia, hunin yuxabu yukaa:
— Yuxabun, min hawa
katsi kini waai? aka.
— En haska waxun, piti kuxi
pitima, xeki patxi besti en piaya,
ea itxabu, huxun, en kini waai
yaix katsidan, aka.
Hunin ninkaa, hawen
dabanen iki, kaxaaya.

Haska wabidani, hukidan,
hawen nabu yuia:
— En nabun, mi hawa
katsi yuxabu
itxa kamen, aka.
— Habia en xeki patxi
ea pianaya,
en itxaa, kaaki, aka.
—Yuxabudan uani kini
waai, en uinbidanxuki,
yaix katsidan, aka.

Hawen bake yuia, kaxaaya.
— Yuxabu ma yaixa. Hanunkain, hinayatan,
pexakayatan, nuxakayatan, buxakayatan.
Haska wakin, keyua. Yuxabu hawen bake
manui: “En bake itannun” ika. “Huu” aka.

Hawen bake hawen ibu
manui, kaxawankainkainaya.
Hawen bake, ha mesti bai
tanai, kaxakukuaya.
Yuxabu kaxai ninkaa:
— En bake kaxaai, uintannun, ika.
Huaya, bake pixta kaxai, tsauken,
bake pixta yaix betxia, benimaaya.
Yaixin bake pixta yuikin:
— En bake, en mia yuai, aka.

Bake pixta benimaai, tsauken.
Hanunkain, yuxabun bake pixta
hawen hiwe medan yukin.
Bake pixta hina waxun, pexaka
waxun, buxaka waxun.
Haska waxun, bake pixta
benimani kiaki.
Yuxabudan eskani kiaki,
yaix katsidan.

Kadi bikindan, yaixin bini kiaki.
Kadimakendan yaixdan
xena besti pimis kiaki.
Yuxabudan eskani kiaki, yaix
katsidan. Hatixunki, yamaki.